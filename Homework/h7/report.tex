\documentclass[11pt,a4paper]{article}
% \usepackage[utf8]{inputenc}
% \usepackage[T1]{fontenc}
\usepackage[english]{babel}
% \usepackage[demo]{graphicx}

% My packages
\usepackage{algorithm, algorithmic, listings} % Code
\usepackage{amsmath, amstext, amssymb, amsfonts, amsthm, dsfont, cancel, gensymb, mathtools, textcomp} % Math
\usepackage{color, xcolor} % Color
\usepackage{diagbox, tabularx} % Table
\usepackage{enumerate} % List
\usepackage{epsfig, epstopdf, graphicx, multicol, multirow, palatino, pgfplots, subcaption, tikz} % Image.
\usepackage{fancybox}
\usepackage{verbatim}

% \usepackage[font=footnotesize]{caption} % labelfont=bf
% \usepackage[font=scriptsize]{subcaption} % labelfont=bf
\usepackage[margin=1in]{geometry}
\usepackage[hidelinks]{hyperref}
\epstopdfsetup{outdir=./Figure/Converted/}
\graphicspath{{./Figure/}}

\makeatletter
\def\input@path{{./Figure/}}
\makeatother

\pgfplotsset{compat=1.13}

\def\BibTeX{{\rm B\kern-.05em{\sc i\kern-.025em b}\kern-.08em
    T\kern-.1667em\lower.7ex\hbox{E}\kern-.125emX}}

\newcommand{\image}[3]{
	\begin{figure}[!ht]
		\centering
	    \includegraphics[width=#1\textwidth]{#2}
		\caption{#3}
		\label{fig:#2}
	\end{figure}
}

\title{
	VE475 Introduction to Cryptography \\
	Homework 7
}
\author{
	Jiang, Sifan\\
	jasperrice@sjtu.edu.cn\\
	515370910040
}


\begin{document}
\maketitle
\section*{Homework 6}
\subsection*{Ex. 5 - Merkle-Damg$\mathrm{\mathbf{\mathring{a}}}$rd construction}
\begin{enumerate}
\item \begin{enumerate}[a)]
\item Since $f(0)=0$ and $f(1)=01$, $f(x_{i})$ is always start with $0$. So $y$ can be separated into several segments start from $0$, except for the first two digits. Those segments are injective with $x_{i}$, so the map $s$ from $x$ to $y$ is injective.

\item If $z$ is empty, from what previous proved, there's no such $x'$. If $z$ is not empty, since we have $11$ at the beginning of $y_{i+1}$, so no this no such $x'$ and $z$ such that $s(x) = z \| s(x')$ .
\end{enumerate}

\item Because the previous conditions guarantee the mapping is collision resistant.
\end{enumerate}



\section*{Homework 7}
\subsection*{Ex. 1 - Cramer-Shoup cryptosystem}
\begin{enumerate}
\item \textbf{Key generation:}
	\begin{itemize}
	\item Alice generates a cyclic group $G$ of order $q$ with two distinct generators $g_{1}$, $g_{2}$. $G$ could be $\mathsf{U}(\mathbb{Z}/p\mathbb{Z})$.
	\item Alice chooses five random values $\left( x_{1}, x_{2}, y_{1}, y_{2}, z \right)$ from $\left\lbrace 0, 1, \cdots, q-1 \right\rbrace$.
	\item Alice computes $c = g_{1}^{x_{1}}g_{2}^{x_{2}}$, $d = g_{1}^{y_{1}}g_{2}^{y_{2}}$, and $h = g_{1}^{z}$.
	\item Alice publishes $\left( c, d, h \right)$ and $G$, $q$, $g_{1}$, $g_{2}$ as her public key. She retains $\left( x_{1}, x_{2}, y_{1}, y_{2}, z \right)$ as her private key.
	\end{itemize}

\par \textbf{Encryption:}
	\begin{itemize}
	\item Bob converts plaintext into an element $m$ in group $G$.
	\item Bob chooses a random $k$ from $\left\lbrace 0, 1, \cdots, q-1 \right\rbrace$, then calculates:
		\begin{itemize}
		\item $u_{1} = g_{1}^{k}$, $u_{2} = g_{2}^{k}$.
		\item $e = h^{k}m$.
		\item $\alpha = H(u_{1}, u_{2}, e)$, where $H$ is a collision-resistant cryptographic hash function.
		\item $v = c^{k}d^{k\alpha}$.
		\end{itemize}
	\item Bob sends the ciphertext $\left( u_{1}, u_{2}, e, v \right)$ to Alice.
	\end{itemize}
	
\par \textbf{Decryption:}
	\begin{itemize}
		\item Alice computes $\alpha = H(u_{1}, u_{2}, e)$ and verifies that 
	\end{itemize}

\item 
\item
\end{enumerate}



\subsection*{Ex. 2 - Simple questions}
\begin{enumerate}
\item
\item
\end{enumerate}



\subsection*{Ex. 3 - Birthday paradox}
\begin{enumerate}
\item
\item
\item
\item
\end{enumerate}



\subsection*{Ex. 4 - Birthday attack}
\begin{enumerate}
\item
\item
\item
\end{enumerate}



\subsection*{Ex. 5 - Faster multiple modular exponentiation}
\begin{enumerate}
\item
\item
\item
\item
\end{enumerate}
\end{document}