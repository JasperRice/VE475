\documentclass[11pt,a4paper]{article}
% \usepackage[utf8]{inputenc}
% \usepackage[T1]{fontenc}
\usepackage[english]{babel}
% \usepackage[demo]{graphicx}

% My packages
\usepackage{algorithm, algorithmic, listings} % Code
\usepackage{amsmath, amstext, amssymb, amsfonts, amsthm, dsfont, cancel, gensymb, mathtools, textcomp} % Math
\usepackage{color, xcolor} % Color
\usepackage{diagbox, tabularx} % Table
\usepackage{enumerate} % List
\usepackage{epsfig, epstopdf, graphicx, multicol, multirow, palatino, pgfplots, subcaption, tikz} % Image.
\usepackage{fancybox}
\usepackage{verbatim}

% \usepackage[font=footnotesize]{caption} % labelfont=bf
% \usepackage[font=scriptsize]{subcaption} % labelfont=bf
\usepackage[margin=1in]{geometry}
\usepackage[hidelinks]{hyperref}
\epstopdfsetup{outdir=./Figure/Converted/}
\graphicspath{{./Figure/}}

\makeatletter
\def\input@path{{./Figure/}}
\makeatother

\pgfplotsset{compat=1.13}

\def\BibTeX{{\rm B\kern-.05em{\sc i\kern-.025em b}\kern-.08em
    T\kern-.1667em\lower.7ex\hbox{E}\kern-.125emX}}

\newcommand{\image}[3]{
	\begin{figure}[!ht]
		\centering
	    \includegraphics[width=#1\textwidth]{#2}
		\caption{#3}
		\label{fig:#2}
	\end{figure}
}

\title{
	VE475 Introduction to Cryptography \\
	Homework 4
}
\author{
	Jiang, Sifan\\
	jasperrice@sjtu.edu.cn\\
	515370910040
}


\begin{document}
\maketitle

\section*{Ex. 1 - Euler's totient}
\begin{enumerate}
\item Notice that for a given prime $p$, we have $\varphi(p) = p-1$. So, positive integers $n$ that is smaller than $p^{k}$, so that $\gcd(n, p^{k}) \neq 1$, can be $1 \times p$, $2 \times p$, $3 \times p$, $\cdots$, $(p^{k-1}-1) \times p$. So the amount of possible $n$ is $p^{k-1}-1$. Also, there are $p_{k}-1$ positive integers are smaller than $p^{k}$, so for any prime $p$, $\varphi(p^{k}) = (p^{k}-1)-(p^{k-1}-1) = p^{k-1}(p-1)$.

\item Since $m$ and $n$ are coprime integers, according to Chinese Remainder theorem, there exists a ring isomorphism between $\mathbb{Z}/mn\mathbb{Z}$ and $\mathbb{Z}/m\mathbb{Z} \times \mathbb{Z}/n\mathbb{Z}$. We have $\varphi(mn)$ is the order of $\mathbb{Z}/mn\mathbb{Z}$, $\varphi(m)$ is the order of $\mathbb{Z}/m\mathbb{Z}$, and $\varphi(n)$ is the order of $\mathbb{Z}/n\mathbb{Z}$. Since an isomorphism is a bijection that preserves algebraic structures, we would have $\varphi(mn) = \varphi(m) \times \varphi(n)$.

\item Assume $n = \prod_{i} p_{i}^{k_{i}}$, then applying the previous results to integer $n > 1$, we have
\begin{align*}
	\varphi(n) =& \prod_{i} \varphi(p_{i}^{k_{i}}) \\
	=& \prod_{i} p_{i}^{k_{i}-1}(p_{i}-1) \\
	=& \prod_{i} p_{i}^{k_{i}}(1-\frac{1}{p_{i}}) \\
	=& n \prod_{p\vert n}(1-\frac{1}{p})
\end{align*}

\item The three last digits of $7^{803}$ can be obtained by calculating $7^{803} \mod 1000$. We note that $1000 = 2^{3} \times 5^{3}$, thus $\varphi(1000) = 1000 \times (1-\frac{1}{2}) \times (1-\frac{1}{5}) = 400$ according to the previous result. So we would have
\begin{align*}
	7^{803} \equiv& (7^{400})^{2}\times 7^{3} \mod 1000 \\
	\equiv& 7^{3} \mod 1000 \\
	\equiv& 7^{3} \mod 1000 \\
	\equiv& 343	\mod 1000
\end{align*}
\par So, the three last digits of $7^{803}$ are $343$.
\end{enumerate}


\newpage
\section*{Ex. 2 - AES}
\begin{enumerate}
\item The key used for round $1$ is given by the columns $K(4)$, $\cdots$, $K(7)$. Also, recall that for $i \not\equiv 0 \mod 4$, $K(i) = K(i-4) \oplus K(i-1)$, and for $i \equiv 0 \mod 4$, $K(i) = K(i-4) \oplus T(K(i-1))$.
\end{enumerate}


\section*{Ex. 3 - Simple questions}
\begin{enumerate}
\item In mode ECB, each block is encrypted independently with a function $E$ and a key $k$, so corruption of one block wouldn't influence other blocks. So the number of plaintext decrypted incorrectly is one for the ECB mode.
\par In mode CBC, after the second block, XOR operation between the previous ciphertext and the current plaintext is first done before the $E$ function and key $k$. So if one block is corrupted, the next block will also be influenced, thus the number of plaintext decrypted incorrectly is two for the CBC mode.

\item Since the length of block is finite, so the $IV$ would be repeated after $2^{n}$ trials, where $n$ is the block length. The attacker then can use whatever plaintext and compare the ciphertext generated with the same $IV$ to find the pattern. In this way, the schemes are not CPA secure.

\item Since $p-1 = 29-1 = 28 = 2\times2\times7$, so $q\in\{2, 7\}$.
	\begin{itemize}
	\item When $q=2$, we have
	\begin{align*}
		2^{(29-1)/2} \equiv 2^{14} \equiv 2^{4}\cdot32^{2} \equiv 2^{4}\cdot3^{2} \equiv 2^{2}\cdot7 \equiv 28 \not\equiv 1 \mod 29
	\end{align*}
	
	\item When $q=7$, we have
	\begin{align*}
		2^{(29-1)/7} \equiv 2^{4} \equiv 16 \not\equiv 1 \mod 29
	\end{align*}
	\end{itemize}
\par So, $2$ is a generator of $\mathsf{U}(\mathbb{Z}/29\mathbb{Z})$.

\item Using proposition from Jacobi symbol, since $1801$ and $8191$ are odd prime positive integers, and $1801 \equiv 1 \mod 4$, we have
\begin{align*}
	\left(\frac{1801}{8191}\right) =& + \left(\frac{8191}{1801}\right) = + \left(\frac{987}{1801}\right) \\
	=& + \left(\frac{1801}{987}\right) = + \left(\frac{814}{987}\right) = + \left(\frac{2\times11\times37}{3\times7\times47}\right) = + \left(\frac{2}{987}\right)\left(\frac{11}{987}\right)\left(\frac{37}{987}\right) \\
	=& + \left(\frac{987}{11}\right)\left(\frac{987}{37}\right) = + \left(\frac{8}{11}\right)\left(\frac{25}{37}\right) = + \left(\frac{2^{3}}{11}\right)\left(\frac{5^{2}}{37}\right) = + \left(\frac{2}{11}\right)^{3}\left(\frac{5}{37}\right)^{2} \\
	=& - \left(\frac{37}{5}\right)^{2} = - \left(\frac{2}{5}\right) = -1
\end{align*}

\item If $\left(\frac{b^{2}-4ac}{p}\right) = 0$, meaning $b^{2}-4ac = 0$, then the equation have one solution $x = -\frac{b}{2a}$ (more technically speaking, two same solutions). Since $-\frac{b}{2a}$ can always mod $p$, it's true that the number of solutions is $1 + \left(\frac{b^{2}-4ac}{p}\right) = 1$.
\par If $\left(\frac{b^{2}-4ac}{p}\right) \neq 0$, meaning $b^{2}-4ac \neq 0$, then the equation have two different solutions $x_{1} = \frac{-b+\sqrt{b^{2}-4ac}}{2a}$ and $x_{2} = \frac{-b-\sqrt{b^{2}-4ac}}{2a}$. So we would have
\begin{align*}
	\frac{-b\pm\sqrt{b^{2}- 4ac}}{2a} \equiv& x \mod p \\
	\sqrt{b^{2}-4ac} \equiv& \pm(2ax+b) \mod p
\end{align*}
\par Then, if $\left(\frac{b^{2}-4ac}{p}\right) = 1$, meaning $b^{2}-4ac$ is a square mod $p$, then it's true that the number of solutions mod $p$ is $1+\left(\frac{b^{2}-4ac}{p}\right) = 2$.
\par Otherwise, if $\left(\frac{b^{2}-4ac}{p}\right) = -1$, meaning $b^{2}-4ac$ is not a square mod $p$, it's true that the number of solutions mod $p$ is $1+\left(\frac{b^{2}-4ac}{p}\right) = 0$.

\item Since $\gcd(n, pq) = 1$, we have $\gcd(n,p) = 1$ and $\gcd(n,q) = 1$. Also, since $q-1$ divides $p-1$, we have $(p-1) = k(q-1)$, where $k$ is a positive integer. So, according to Euler's theorem, we have
\begin{align*}
	n^{p-1} \equiv& 1 \mod p \\
	(n^{q-1})^{k} \equiv n^{p-1} \equiv& 1 \mod q
\end{align*}
\par Since $\gcd(n^{p-1}, p) = 1$ and $\gcd(n^{p-1}, q) = 1$, we can conclude that $\gcd(n^{p-1}, pq) = 1$, which is
\begin{align*}
	n^{p-1} \equiv 1 \mod pq
\end{align*}

\item \begin{itemize}
\item Sufficiency: if $p \equiv 1 \mod 3$, then we can obtain
\begin{align*}
	\left(\frac{p}{3}\right) = 1
\end{align*}
\par Also, note that $p$ is an odd prime. If $p \equiv 1 \mod 4$
\begin{align*}
	\left(\frac{-3}{p}\right) =& \left(\frac{-1}{p}\right)\left(\frac{3}{p}\right) = 1 \cdot 1 = 1
\end{align*}
\par If $p \equiv 3 \mod 4$
\begin{align*}
	\left(\frac{-3}{p}\right) =& \left(\frac{-1}{p}\right)\left(\frac{3}{p}\right) = (-1) \cdot (-1) = 1
\end{align*}

\item Necessity: We already known $\left(\frac{-3}{p}\right) = \left(\frac{-1}{p}\right)\left(\frac{3}{p}\right) = 1$, since $p$ is an odd prime, $p \equiv 1 \mod 4$ or $p \equiv 3 \mod 4$.
\par If $p \equiv 1 \mod 4$, $\left(\frac{-1}{p}\right) = 1$, thus $\left(\frac{3}{p}\right) = \left(\frac{p}{3}\right) = 1$.
\par If $p \equiv 3 \mod 4$, $\left(\frac{-1}{p}\right) = -1$, thus $\left(\frac{3}{p}\right) = -\left(\frac{p}{3}\right) = -1$.
\par So, in both case $\left(\frac{p}{3}\right) = 1$, which gives $p^{(3-1)/2} \equiv p \equiv 1 \mod 3$.
\end{itemize}

\item If $\left(\frac{a}{p}\right) = 1$, we would have $a^{(p-1)/2} \equiv 1 \mod p$. However $2$ is a prime factor of $p-1$, meaning $2 \vert (p-1)$. So conflict with the requirement to be a generator, thus $a$ is not a generator of $\mathbb{F}_{p}^{*}$.
\end{enumerate}


\section*{Ex. 4 - Prime vs. irreducible}
\begin{enumerate}
\item 
\end{enumerate}


\section*{Ex. 5 - Primitive root mod 65537}
\begin{enumerate}
\item Using proposition of Jacobi symbol, since $65537 \equiv 1 \mod 4$ and $\gcd(3, 65537) = 1$, we have
\begin{align*}
	\left(\frac{3}{65537}\right) =& + \left(\frac{65537}{3}\right) \\
	=& + \left(\frac{2}{3}\right) \\
	=& - 1
\end{align*}
\par Meaning $3$ is not a square mod $65537$.

\item Since $65537$ is a prime integer, $\frac{65537-1}{2} = 32768$, and $3$ is not a square mod $65537$, we can conclude that $3^{32768}\not\equiv 1 \mod 65537$.

\item First note that $p-1=65537-1 = 2^{16}$, thus $q=2$ is the only prime such that $q \vert (p-1)$. Also, since $3^{32768}\equiv \alpha^{(p-1)/q} \not\equiv 1 \mod p$, and according to the theorem, we can conclude that $\alpha = 3$ is a primitive root mod $65537$.
\end{enumerate}
\end{document}