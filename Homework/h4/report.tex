\documentclass[11pt,a4paper]{article}
% \usepackage[utf8]{inputenc}
% \usepackage[T1]{fontenc}
\usepackage[english]{babel}
% \usepackage[demo]{graphicx}

% My packages
\usepackage{algorithm, algorithmic, listings} % Code
\usepackage{amsmath, amstext, amssymb, amsfonts, amsthm, dsfont, cancel, gensymb, mathtools, textcomp} % Math
\usepackage{color, xcolor} % Color
\usepackage{diagbox, tabularx} % Table
\usepackage{enumerate} % List
\usepackage{epsfig, epstopdf, graphicx, multicol, multirow, palatino, pgfplots, subcaption, tikz} % Image.
\usepackage{fancybox}
\usepackage{verbatim}

% \usepackage[font=footnotesize]{caption} % labelfont=bf
% \usepackage[font=scriptsize]{subcaption} % labelfont=bf
\usepackage[margin=1in]{geometry}
\usepackage[hidelinks]{hyperref}
\epstopdfsetup{outdir=./Figure/Converted/}
\graphicspath{{./Figure/}}

\makeatletter
\def\input@path{{./Figure/}}
\makeatother

\pgfplotsset{compat=1.13}

\def\BibTeX{{\rm B\kern-.05em{\sc i\kern-.025em b}\kern-.08em
    T\kern-.1667em\lower.7ex\hbox{E}\kern-.125emX}}

\newcommand{\image}[3]{
	\begin{figure}[!ht]
		\centering
	    \includegraphics[width=#1\textwidth]{#2}
		\caption{#3}
		\label{fig:#2}
	\end{figure}
}

\title{
	VE475 Introduction to Cryptography \\
	Homework 4
}
\author{
	Jiang, Sifan\\
	jasperrice@sjtu.edu.cn\\
	515370910040
}


\begin{document}
\maketitle

\section*{Ex. 1 - Euler's totient}
\begin{enumerate}
\item Notice that for a given prime $p$, we have $\varphi(p) = p-1$. So, positive integers $n$ that is smaller than $p^{k}$, so that $\gcd(n, p^{k}) \neq 1$, can be $1 \times p$, $2 \times p$, $3 \times p$, $\cdots$, $(p^{k-1}-1) \times p$. So the amount of possible $n$ is $p^{k-1}-1$. Also, there are $p_{k}-1$ positive integers are smaller than $p^{k}$, so for any prime $p$, $\varphi(p^{k}) = (p^{k}-1)-(p^{k-1}-1) = p^{k-1}(p-1)$.

\item Since $m$ and $n$ are coprime integers, according to Chinese Remainder theorem, there exists a ring isomorphism between $\mathbb{Z}/mn\mathbb{Z}$ and $\mathbb{Z}/m\mathbb{Z} \times \mathbb{Z}/n\mathbb{Z}$. We have $\varphi(mn)$ is the order of $\mathbb{Z}/mn\mathbb{Z}$, $\varphi(m)$ is the order of $\mathbb{Z}/m\mathbb{Z}$, and $\varphi(n)$ is the order of $\mathbb{Z}/n\mathbb{Z}$. Since an isomorphism is a bijection that preserves algebraic structures, we would have $\varphi(mn) = \varphi(m) \times \varphi(n)$.

\item Assume $n = \prod_{i} p_{i}^{k_{i}}$, then applying the previous results to integer $n > 1$, we have
\begin{align*}
	\varphi(n) =& \prod_{i} \varphi(p_{i}^{k_{i}}) \\
	=& \prod_{i} p_{i}^{k_{i}-1}(p_{i}-1) \\
	=& \prod_{i} p_{i}^{k_{i}}(1-\frac{1}{p_{i}}) \\
	=& n \prod_{p\vert n}(1-\frac{1}{p})
\end{align*}

\item The three last digits of $7^{803}$ can be obtained by calculating $7^{803} \mod 1000$. We note that $1000 = 2^{3} \times 5^{3}$, thus $\varphi(1000) = 1000 \times (1-\frac{1}{2}) \times (1-\frac{1}{5}) = 400$ according to the previous result. So we would have
\begin{align*}
	7^{803} \equiv& (7^{400})^{2}\times 7^{3} \mod 1000 \\
	\equiv& 7^{3} \mod 1000 \\
	\equiv& 7^{3} \mod 1000 \\
	\equiv& 343	\mod 1000
\end{align*}
\par So, the three last digits of $7^{803}$ are $343$.
\end{enumerate}


\newpage
\section*{Ex. 2 - AES}
\begin{enumerate}
\item The key used for round $1$ is given by the columns $K(4)$, $\cdots$, $K(7)$. Also, recall that for $i \not\equiv 0 \mod 4$, $K(i) = K(i-4) \oplus K(i-1)$, and for $i \equiv 0 \mod 4$, $K(i) = K(i-4) \oplus T(K(i-1))$.
\end{enumerate}


\section*{Ex. 3 - Simple questions}


\section*{Ex. 4 - Prime vs. irreducible}


\section*{Ex. 5 - Primitive root mod 65537}
\begin{enumerate}
\item Using proposition of Jacobi symbol, since $65537 \equiv 1 \mod 4$ and $\gcd(3, 65537) = 1$, we have
\begin{align*}
	\left(\frac{3}{65537}\right) =& +\left(\frac{65537}{3}\right) \\
	=& +\left(\frac{2}{3}\right) \\
	=& -1
\end{align*}
\par Meaning $3$ is not a square mod $65537$.

\item Since $65537$ is a prime integer, $\frac{65537-1}{2} = 32768$, and $3$ is not a square mod $65537$, we can conclude that $3^{32768}\not\equiv 1 \mod 65537$.

\item First note that $p-1=65537-1 = 2^{16}$, thus $q=2$ is the only prime such that $q \vert (p-1)$. Also, since $3^{32768}\equiv \alpha^{(p-1)/q} \not\equiv 1 \mod p$, and according to the theorem, we can conclude that $\alpha = 3$ is a primitive root mod $65537$.
\end{enumerate}
\end{document}