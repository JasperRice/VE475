\documentclass[11pt,a4paper]{article}
% \usepackage[utf8]{inputenc}
% \usepackage[T1]{fontenc}
\usepackage[english]{babel}
% \usepackage[demo]{graphicx}

% My packages
\usepackage{algorithm, algorithmic, listings} % Code
\usepackage{amsmath, amstext, amssymb, amsfonts, amsthm, dsfont, cancel, gensymb, mathtools, textcomp} % Math
\usepackage{color, xcolor} % Color
\usepackage{diagbox, tabularx} % Table
\usepackage{enumerate} % List
\usepackage{epsfig, epstopdf, graphicx, multicol, multirow, palatino, pgfplots, subcaption, tikz} % Image.
\usepackage{fancybox}
\usepackage{verbatim}

% \usepackage[font=footnotesize]{caption} % labelfont=bf
% \usepackage[font=scriptsize]{subcaption} % labelfont=bf
\usepackage[margin=1in]{geometry}
\usepackage[hidelinks]{hyperref}
\epstopdfsetup{outdir=./Figure/Converted/}
\graphicspath{{./Figure/}}

\makeatletter
\def\input@path{{./Figure/}}
\makeatother

\pgfplotsset{compat=1.13}

\def\BibTeX{{\rm B\kern-.05em{\sc i\kern-.025em b}\kern-.08em
    T\kern-.1667em\lower.7ex\hbox{E}\kern-.125emX}}

\newcommand{\image}[3]{
	\begin{figure}[!ht]
		\centering
	    \includegraphics[width=#1\textwidth]{#2}
		\caption{#3}
		\label{fig:#2}
	\end{figure}
}

\title{
	VE475 Introduction to Cryptography \\
	Homework 3
}
\author{
	Jiang, Sifan\\
	jasperrice@sjtu.edu.cn\\
	515370910040
}


\begin{document}
\maketitle


\section*{Ex. 1 - Finite fields}
\begin{enumerate}
	\item 

	\item 
	
	\item 
\end{enumerate}


\section*{Ex. 2 - Rabin cryptosystem}
\begin{enumerate}
	\item The Rabin cryptosystem uses a private key and a public key at the same time. The system works as followed.
	\par First, choose two large different primes $p$ and $q$. The primes $p$ and $q$ are the private key and let $n = pq$ be the public key. The public key is used in the encryption while the private key is required in the decryption.
	\par Then in the encryption part, let $m \in \{ 0, \cdots, n-1 \}$ be the plaintext. And the ciphertext $c$ is determined by
		\begin{align*}
			c &= m^{2} \mod n
		\end{align*}
	\par And for most of the ciphertexts, there are exactly four possible plaintexts could lead to the same ciphertext.
	\par To efficiently decrypt the ciphertext, the private key is necessary. We can use the Chinese remainder theorem to solve for $m$. We have to calculate the square roots (will be explained)
		\begin{align*}
			m_{p} &= \sqrt{c} \mod p \\
			m_{q} &= \sqrt{c} \mod q
		\end{align*}
	\par After get the value of $m_{p}$ and $m_{q}$, we then apply the extended Euclidean algorithm to find $y_{p}$ and $y_{q}$ such that $y_{p}\cdot p + y_{q}\cdot q = 1$. Then by using the Chinese remainder theorem, the four square roots $r$, $-r$, $s$, and $-s$ can be calculated
        \begin{align*}
            r &= (y_{p} \cdot p \cdot m_{q} + y_{q} \cdot q \cdot m_{p}) \mod n \\
            -r &= n - r \\
            s &= (y_{p} \cdot p \cdot m_{q} - y_{q} \cdot q \cdot m_{p}) \mod n \\
            -s &= n - s
        \end{align*}
    \par Then $m \in \{ r, -r, s, -s \}$.
    \par To simplify the computation of $m_{p} = \sqrt{c} \mod p$ and $m_{q} = \sqrt{c} \mod q$, we can choose $p \equiv q \equiv 3 \mod 4$ and get square roots by calculating
        \begin{align*}
            m_{p} &= c^{\frac{1}{4}(p+1)} \mod p \\
            m_{q} &= c^{\frac{1}{4}(q+1)} \mod q
        \end{align*}

    \item
    \begin{enumerate}[a)]
        \item As explained, there are at most $4$ possible results of $m = \sqrt{x} \mod n$, then the probability of getting a meaningful message is at least $25\%$. So within few trials, the probability of getting this message is fairly high.

        \item It won't be easy for Eve to break the ciphertext. After getting the ciphertext $x$ and the public key $n$, she directly solve $m = \sqrt{x} \mod n$ or factorized $n$ to get $p$ and $q$. However, there's no effective way to solve $m = \sqrt{x} \mod n$ or solving the factorization of $n$ which is the product of two large prime number.

        \item She can use Chosen Ciphertext Attack (CCA). Since she has stolen the device, she can get the four outputs of an arbitrary input $x$. And based on the public key $n$, she can determined the $r$, $-r$, $s$, and $-s$. Then $\gcd(r-s, n)$ is a factor of $n$. Then she can find $p$ and $q$.
    \end{enumerate}

\end{enumerate}

\section*{Ex. 3 - CRT}
\par Assume there are at least $x$ people in the group, we then have
	\begin{align*}
		x &\equiv 1 \mod 3 \\
		x &\equiv 2 \mod 4 \\
		x &\equiv 3 \mod 5
	\end{align*}
\par To solve $x$, we need to apply the Chinese remainder theorem
	\begin{itemize}
		\item Step 1:
			\begin{align*}
				&\mbox{Common\_multiple}(4, 5) \equiv 40 \equiv 1 \mod 3 \\
				&\mbox{Common\_multiple}(5, 3) \equiv 45 \equiv 1 \mod 4 \\
				&\mbox{Common\_multiple}(3, 4) \equiv 36 \equiv 1 \mod 5
			\end{align*}

		\item Step 2:
			\begin{align*}
				40 \times 1 &= 40 \\
				45 \times 2 &= 90 \\
				36 \times 3 &= 108 \\
				40 + 90 + 108 &= 238
			\end{align*}

		\item Step 3:
			\begin{align*}
				x &\equiv 238 \mod \mbox{Lowest\_common\_multiple} (3, 4 , 5) \\
				&\equiv 58 \mod 60 \\
				&\equiv 118 \mod 60
			\end{align*}
	\end{itemize}
\par So the two smallest possible number of people in the group are $58$ and $118$.

\end{document}