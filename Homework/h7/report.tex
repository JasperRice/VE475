\documentclass[11pt,a4paper]{article}
% \usepackage[utf8]{inputenc}
% \usepackage[T1]{fontenc}
\usepackage[english]{babel}
% \usepackage[demo]{graphicx}

% My packages
\usepackage{algorithm, algorithmicx, algpseudocode} % Algorithm {algorithmic, listings}
\usepackage{amsmath, amstext, amssymb, amsfonts, amsthm, dsfont, cancel, gensymb, mathtools, textcomp} % Math
\usepackage{color, xcolor} % Color
\usepackage{diagbox, tabularx} % Table
\usepackage{enumerate} % List
\usepackage{epsfig, epstopdf, graphicx, multicol, multirow, palatino, pgfplots, subcaption, tikz} % Image.
\usepackage{fancybox}
\usepackage{verbatim}

% \usepackage[font=footnotesize]{caption} % labelfont=bf
% \usepackage[font=scriptsize]{subcaption} % labelfont=bf
\usepackage[margin=1in]{geometry}
\usepackage[hidelinks]{hyperref}
\epstopdfsetup{outdir=./Figure/Converted/}
\graphicspath{{./Figure/}}

\makeatletter
\def\input@path{{./Figure/}}
\makeatother

\newtheorem{theorem}{Theorem}
\newtheorem{definition}{Definition}
\floatname{algorithm}{Algorithm}
\renewcommand{\algorithmicrequire}{\textbf{Input:}}
\renewcommand{\algorithmicensure}{\textbf{Output:}}

\pgfplotsset{compat=1.13}

\newcommand{\image}[3]{
	\begin{figure}[!ht]
		\centering
	    \includegraphics[width=#1\textwidth]{#2}
		\caption{#3}
		\label{fig:#2}
	\end{figure}
}

\title{
	VE475 Introduction to Cryptography \\
	Homework 7
}
\author{
	Jiang, Sifan\\
	jasperrice@sjtu.edu.cn\\
	515370910040
}


\begin{document}
\maketitle
\section*{Homework 6}
\subsection*{Ex. 5 - Merkle-Damg$\mathrm{\mathbf{\mathring{a}}}$rd construction}
\begin{enumerate}
\item \begin{enumerate}[a)]
\item Since $f(0)=0$ and $f(1)=01$, $f(x_{i})$ is always start with $0$. So $y$ can be separated into several segments start from $0$, except for the first two digits. Those segments are injective with $x_{i}$, so the map $s$ from $x$ to $y$ is injective.

\item If $z$ is empty, from what previous proved, there's no such $x'$. If $z$ is not empty, since we have $11$ at the beginning of $y_{i+1}$, so there's no $x'$ and $z$ such that $s(x) = z \| s(x')$.
\end{enumerate}

\item Because the previous conditions guarantee the mapping is collision resistant.

\item Assuming we have a collision on $h$, i.e. $x \neq x'$ and $h(x) = h(x')$, we will prove that a collision on the compression function $g$ can be efficiently found.
\par First note that if $x \neq x'$, since the map $s$ from $x$ to $y$ is injective, it would always lead to $y \neq y'$. Let $k$ and $k'$ denote the number blocks for $y$ and $y'$.
\par \textbf{Case 1:} consider $k = k'$. This implies $y_{k} = y_{k'}$, and we have
	\begin{align*}
	g(z_{k-1}\|y_{k}) &= z_{k} = h(x) \\
	&= h(x') = z_{k'} \\
	&= g(z_{k'-1}\|y_{k'})
	\end{align*}
\par If $z_{k} \neq z_{k'}$, then a collision is found. Otherwise we repeat the process and get
	\begin{align*}
	g(z_{k-2}\|y_{k-1}) &= z_{k-1} = h(x) \\
	&= h(x') = z_{k'-1} \\
	&= g(z_{k'-2}\|y_{k'-1})
	\end{align*}
\par Then either we have found a collision or we continue backward until one is obtained. If none is found then we get $z_{i} = z_{i'}$, where $i \in [1, k]$.
\par \textbf{Case 2:} consider $k \neq k'$. Without loss of generality assume $k'$ and proceed as in case $1$. If no collision is found before $k = 1$, then we have
	\begin{align*}
	g(0^{m}\|y_{1}) &= z_{1} \\
	&= z_{k'-k+1} \\
	&= g(z_{k'-k}\|y_{k'-k+1})
	\end{align*}
\par Since there is not strings $x \neq x'$ and $z$ such that $s(x) = z \| s(x')$, the collision is found.
\end{enumerate}



\section*{Homework 7}
\subsection*{Ex. 1 - Cramer-Shoup cryptosystem}
\begin{enumerate}
\item \textbf{Key generation:}
	\begin{itemize}
	\item Alice generates a cyclic group $G$ of order $q$ with two distinct generators $g_{1}$, $g_{2}$. $G$ could be $\mathsf{U}(\mathbb{Z}/p\mathbb{Z})$.
	\item Alice chooses five random values $\left( x_{1}, x_{2}, y_{1}, y_{2}, z \right)$ from $\left\lbrace 0, 1, \cdots, q-1 \right\rbrace$.
	\item Alice computes $c = g_{1}^{x_{1}}g_{2}^{x_{2}}$, $d = g_{1}^{y_{1}}g_{2}^{y_{2}}$, and $h = g_{1}^{z}$.
	\item Alice publishes $\left( c, d, h \right)$ and $G$, $q$, $g_{1}$, $g_{2}$ as her public key. She retains $\left( x_{1}, x_{2}, y_{1}, y_{2}, z \right)$ as her private key.
	\end{itemize}
\par \textbf{Encryption:}
	\begin{itemize}
	\item Bob converts plaintext into an element $m$ in group $G$.
	\item Bob chooses a random $k$ from $\left\lbrace 0, 1, \cdots, q-1 \right\rbrace$, then calculates:
		\begin{itemize}
		\item $u_{1} = g_{1}^{k}$, $u_{2} = g_{2}^{k}$.
		\item $e = h^{k}m$.
		\item $\alpha = H(u_{1}, u_{2}, e)$, where $H$ is a collision-resistant cryptographic hash function.
		\item $v = c^{k}d^{k\alpha}$.
		\end{itemize}
	\item Bob sends the ciphertext $\left( u_{1}, u_{2}, e, v \right)$ to Alice.
	\end{itemize}
\par \textbf{Decryption:}
	\begin{itemize}
		\item Alice computes $\alpha = H(u_{1}, u_{2}, e)$ and verifies that $u_{1}^{x_{1}}u_{2}^{x_{2}}(u_{1}^{y_{1}}u_{2}^{y_{2}})^{\alpha} = v$. If it fails, further decryption is aborted.
		\item Since $u_{1}^{z} = g_{1}^{kz} = h^{k}$, and $m = \frac{e}{h^{k}}$, Alice computes the plaintext as $m = \frac{e}{u_{1}^{z}}$.
	\end{itemize}

\item Adaptive chosen ciphertext attacks is an iterative chosen ciphertext attack scenario in which the attacker gradually reveal information about an ciphertext $c$ or private key by iteratively sending new ciphertexts $c'$, $c''$, $\cdots$ that are related to the original ciphertext $c$ to the receiver and analysis the response. However, in the decryption stage of Cramer-Shoup cryptosystem, there's a verification stage where invalid ciphertexts would be rejected. Also, since $H$ is a collision-resistant cryptographic hash function, it's practically infeasible to find enough chosen ciphertext to attack.

\item \begin{itemize}
	\item \textbf{Similarities:} Both cryptosystems are based on the DLP in a cyclic group.
	
	\item \textbf{Differences:} Cramer-Shoup cryptosystem uses a collision-resistant cryptographic hash function for verification to avoid adaptive chosen ciphertext attacks, while the Elgamal cryptosystem doesn't.
	\end{itemize}
\end{enumerate}



\subsection*{Ex. 2 - Simple questions}
\begin{enumerate}
\item According to Fermat's little theorem, if $p$ is prime and $p \nmid \alpha$, then $a^{p-1} \equiv 1 \mod p$. $h(x)$ is not second pre-image resistant because given $x$, it is easy to find $x' = x + p - 1$ such that $h(x) = h(x')$. So, it is not a good cryptographic hash function.

\item We would have
	\begin{align*}
	&\left\lfloor 2^{30}\sqrt{2} \right\rfloor = \left\lfloor 40000000‬\sqrt{2} \right\rfloor = 5A827999 &= K_{i}, \quad &\mbox{where } 0 \leq i \leq 19 \\
	&\left\lfloor 2^{30}\sqrt{3} \right\rfloor = \left\lfloor 40000000‬\sqrt{3} \right\rfloor = 6ED9EBA1 &= K_{i}, \quad &\mbox{where } 20 \leq i \leq 39 \\
	&\left\lfloor 2^{30}\sqrt{4} \right\rfloor = \left\lfloor 40000000‬\sqrt{4} \right\rfloor = 8F1BBCDC &= K_{i}, \quad &\mbox{where } 40 \leq i \leq 59 \\
	&\left\lfloor 2^{30}\sqrt{5} \right\rfloor = \left\lfloor 40000000‬\sqrt{5} \right\rfloor = CA62C1D6 &= K_{i}, \quad &\mbox{where } 60 \leq i \leq 79
	\end{align*}
\end{enumerate}



\subsection*{Ex. 3 - Birthday paradox}
\begin{enumerate}
\item Since $g(x) = \ln(1-x) + x + x^{2}$, we have
	\begin{align*}
	\frac{d g(x)}{d x} = -\frac{1}{1-x} + 1 + 2x
	\end{align*}
\par When $\frac{dg(x)}{dx} = 0$, we have $x_{1} = 0$ and $x_{2} = \frac{1}{2}$. Also, since
	\begin{align*}
	\frac{d^{2} g(x)}{d x^{2}} &= -\frac{1}{(1-x)^{2}} + 2 \\
	\left.\frac{d^{2} g(x)}{d x^{2}}\right\vert_{x = 0} &= 1 > 0 \\
	\left.\frac{d^{2} g(x)}{d x^{2}}\right\vert_{x = \frac{1}{2}} &= -2 < 0 \\
	\end{align*}
we could conclude that for $x \in \left[0, \frac{1}{2}\right]$, $g(x) \in \left[g(0), g\left(\frac{1}{2}\right)\right]$. So $g(x) \geq g(0) = 0$, which gives $-x-x^{2} \leq \ln(1-x)$.
\par Then having $h(x) = \ln(1-x) + x$, we can apply similar method and finally get $\ln(1-x) \leq -x$.
\par In all, when $x \in \left[0, \frac{1}{2}\right]$, $-x-x^{2} \leq \ln(1-x) \leq -x$.

\item Since $j \in [1, r-1]$ and $r \leq \frac{n}{2}$, we would have $\frac{j}{n} \in \left[0, \frac{1}{2}\right]$, thus, according to the result from previous problem
	\begin{align*}
	-\frac{j}{n} - \left(\frac{j}{n}\right)^{2} \leq \ln(1 - \frac{j}{n}) \leq -\frac{j}{n}
	\end{align*}
\par Then, since $r > 1$, apply the sum to each parts,
	\begin{align*}
	\sum_{j=1}^{r-1}\left[-\frac{j}{n} - \left(\frac{j}{n}\right)^{2}\right] &= -\frac{(r-1)r}{2n} - \frac{r(r-1)(2r-1)}{6n^{2}} > -\frac{(r-1)r}{2n} - \frac{r^{3}}{3n^{2}} \\
	\sum_{j=1}^{r-1}\left[-\frac{j}{n}\right] &= -\frac{(r-1)r}{2n}
	\end{align*}
\par So, in all, we would have
	\begin{align*}
	-\frac{(r-1)r}{2n} - \frac{r^{3}}{3n^{2}} \leq \sum_{j=1}^{r-1}\ln\left(1-\frac{j}{n}\right) \leq -\frac{(r-1)r}{2n}
	\end{align*}

\item Apply exponentiate to the previous inequality equation, we would have
	\begin{align*}
	\exp\left(-\frac{(r-1)r}{2n}-\frac{r^3}{3n^2}\right) \leq \prod_{j=1}^{r-1}\left(1-\frac{j}{n}\right) \leq \exp\left(-\frac{(r-1)r}{2n}\right)
	\end{align*}
\par Let $\lambda = \frac{r^2}{2n}$, $c_{1} = \sqrt{\frac{\lambda}{2}}$, and $c_{2} = \sqrt{\frac{\lambda}{2}}$, we would have
	\begin{align*}
	e^{-\lambda}e^{c_{1}/\sqrt{n}} \leq \prod_{j=1}^{r-1} \left(1-\frac{j}{n}\right) \leq e^{-\lambda}e^{c_{2}/\sqrt{n}}
	\end{align*}

\item When $\lambda < \frac{n}{8}$, we would have
	\begin{align*}
	\lambda = \frac{r^2}{2n} < \frac{n}{8}
	\end{align*}
which gives $r < \frac{n}{2}$. Also, when $n$ is large,
	\begin{align*}
	\lim_{n\to\infty}e^{c_1/\sqrt{n}} &= \lim_{n\to\infty}e^0 = 1 \\
	\lim_{n\to\infty}e^{c_2/\sqrt{n}} &= \lim_{n\to\infty}e^0=1
	\end{align*}
\par So, we would have
	\begin{align*}
	\prod_{j=1}^{r-1}\left(1-\frac{j}{n}\right) \approx e^{-\lambda}
	\end{align*}
\end{enumerate}



\subsection*{Ex. 4 - Birthday attack}
\begin{enumerate}
\item The probability of seeing two plates ending with the same three digits is calculated as
	\begin{align*}
	P = 1 - \prod_{j=1}^{39}\left(1 - \frac{j}{1000}\right) \approx 0.54637
	\end{align*}

\item The probability that one of the $40$ license plates observed has the same $3$ last digits is calculated as
	\begin{align*}
	P = 40\left(\frac{1}{1000}\right)\left(\frac{999}{1000}\right)^{39} \approx 0.038469
	\end{align*}

\item From question $1$, we can tell that it's not collision resistant, but from question $2$, we can tell that it's second pre-image resistant. So Alice can prevent the collision by changing the message from Eve a little bit.
\end{enumerate}



\subsection*{Ex. 5 - Faster multiple modular exponentiation}
\begin{enumerate}
\item The time complexity to compute $\alpha^{a} \mod n$ is $\mathcal{O}\left((\log n)^{2}\log a\right)$. The time complexity to compute $\beta^{b} \mod n$ is $\mathcal{O}\left((\log n)^{2}\log b\right)$. So the time complexity of this method is $\mathcal{O}\left((\log n)^{2}(\log a + \log b)\right)$.

\item The algorithm is shown in alg~\ref{alg:FMME}.
	\begin{algorithm}[!ht]
		\caption{Faster Multiple Modular Exponentiation}
		\label{alg:FMME}
		\begin{algorithmic}[1]
			\Require $\alpha$, $\beta$, $a = (a_{k_{a}-1}\cdots a_{0})_{2}$, $b = (b_{k_{b}-1}\cdots b_{0})_{2}$, and $n$ five integers
			\Ensure $\alpha^{a}\beta^{b} \mod n$
			\State $k \leftarrow \max(k_{a}, k_{b})$
			\State $power \leftarrow 1$	
			\For{$i \leftarrow k-1$ \textbf{to} $0$}
				\State $power \leftarrow (power\cdot power) \mod n$
				\If{$a_{i} = 1$}
					\State $power \leftarrow (\alpha \cdot power) \mod n$
				\EndIf
				\If{$b_{i} = 1$}
					\State $power \leftarrow (\beta \cdot power) \mod n$
				\EndIf
			\EndFor
			\State \Return{$power$}
		\end{algorithmic}
	\end{algorithm}
	
\item If $a$ and $b$ are $l$ bits long, $l$ squaring and multiplications are necessary to compute $\alpha^{a}\beta^{b} \mod n$.
\end{enumerate}
\end{document}