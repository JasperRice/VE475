\documentclass[11pt,a4paper]{article}
% \usepackage[utf8]{inputenc}
% \usepackage[T1]{fontenc}
\usepackage[english]{babel}
% \usepackage[demo]{graphicx}

% My packages
\usepackage{algorithm, algorithmic, listings} % Code
\usepackage{amsmath, amstext, amssymb, amsfonts, amsthm, dsfont, cancel, gensymb, mathtools, textcomp} % Math
\usepackage{color, xcolor} % Color
\usepackage{diagbox, tabularx} % Table
\usepackage{enumerate} % List
\usepackage{epsfig, epstopdf, graphicx, multicol, multirow, palatino, pgfplots, subcaption, tikz} % Image.
\usepackage{fancybox}
\usepackage{verbatim}

\usepackage[font=footnotesize]{caption} % labelfont=bf
\usepackage[font=scriptsize]{subcaption} % labelfont=bf
\usepackage[margin=1in]{geometry}
\usepackage[hidelinks]{hyperref}
\epstopdfsetup{outdir=./Figure/Converted/}
\graphicspath{{./Figure/}}

\makeatletter
\def\input@path{{./Figure/}}
\makeatother

\pgfplotsset{compat=1.13}

\def\BibTeX{{\rm B\kern-.05em{\sc i\kern-.025em b}\kern-.08em
    T\kern-.1667em\lower.7ex\hbox{E}\kern-.125emX}}
    
\newcommand{\image}[3]{
	\begin{figure}[!ht]
		\centering
	    \includegraphics[width=#1\textwidth]{#2}
		\caption{#3}
		\label{fig:#2}
	\end{figure}
}

\title{	
	VE475 Introduction to Cryptography \\
	Homework 2
}
\author{
	Jiang, Sifan\\
	jasperrice@sjtu.edu.cn\\
	515370910040
}


\begin{document}
\maketitle

\section*{Ex. 1 - Simple questions}
\begin{enumerate}
	\item The inverse of $17$ modulo $101$ can be found by the extended Euclidean algorithm. Initially, $s_{0} = 0$, $s_{1} = 1$, $t_{0} = 1$, and $t_{1} = 0$.
		\begin{table}[!ht]
			\centering
			\begin{tabular}{lllll}
				$101 = 5 \times 17 + 16$ & $s_{0} = 1$ & $s_{1} = 0$ & $t_{0} = 0$ & $t_{1} = 1$ \\
				$17 = 1 \times 16 + 1$ & $s_{0} = -5$ & $s_{1} = 1$ & $t_{0} = 1$ & $t_{1} = 0$ \\
				$16 = 16 \times 1 + 0$ & $s_{0} = 6$ & $s_{1} = -5$ & $t_{0} = -1$ & $t_{1} = 1$ \\
				$1 = 0 + 1$ & $s_{0} = -101$ & $s_{1} = 6$ & $t_{0} = 17$ & $t_{1} = -1$ \\
			\end{tabular}
		\end{table}
	\par So, we can see that $\gcd(17, 101) = 1$ and the multiplicative inverse of $17$ modulo $101$ is $s_{1} = 6$.
	
	\item Simplify the condition given, we would have
		\begin{align*}
			12x &\equiv 28 \mod 236 \\
			 3x &\equiv 7  \mod 59
		\end{align*}
	\par So, we would have
		\begin{align*}
			3x &= \left\lbrace \begin{array}{l} 59 \cdot (3k + 0) + 7 \\ 59 \cdot (3k + 1) + 7 \\ 59 \cdot (3k + 2) + 7 \end{array} \right., \quad \mbox{where } k \in Z \\
			x &= \left\lbrace \begin{array}{l} 59k + 2 + \frac{1}{3} \\ 59k + 22 \\ 59k + 41 +  \frac{2}{3} \end{array} \right.
		\end{align*}
	\par Since $x \in Z$, $x = 59k + 22$, where $k \in Z$.
	
	\item ?????
	
	\item Since $4883 < 70^{2}$and $4369 < 67^{2}$, the smallest prime factor should be found from: $2$, $3$, $5$, $7$, $11$, $13$, $17$, $19$, $23$, $29$, $31$,, $37$, $41$, $43$, $47$, $53$, $59$, $61$, and $67$. So, we would have $4883 = 19 \times 257$. Since $19$ is the smallest factor of $4883$ and $257 < 17^{2}$, we can conclude that $257$ is also a prime. Similarly, we would also have $4369 = 17 \times 257$, where $257$ is also a prime. In conclusion, we have
		\begin{align*}
			4883 &= 19 \times 257 \\
			4369 &= 17 \times 257
		\end{align*}
		
	\item Assume the matrix $A$ such that
		\begin{align*}
			A = \begin{pmatrix} 3 & 5 \\ 7 & 3 \end{pmatrix} \mod p
		\end{align*}
		is not invertible.
	\par Since $\det(A) = -26$, we need to find prime $p$ such that $\gcd(-26, p) \neq 1$. Or in another word, we need to find primes which are not coprime of $-26$. And since $\vert -26 \vert = 2 \times 13$, we would have $p = 2$ or $p = 13$.
	
	\item Since $ab \equiv 0 \mod p$, we have $ab = kp$, where $k \in Z$. Since $p$ is a prime, we can assume that $\gcd(a, p) = 1$ or $\gcd(a, p) = p$. And when $\gcd(a, p) = p$, $a$ is congruent to $0 \mod p$.
	\par If $\gcd(a, p) = 1$, since $p \vert ab$, we would have $p \vert b$, which means $b$ is congruent to $0 \mod p$.
	\par So, in conclusion, either $a$ or $b$ is congruent to $0 \mod p$.
	
	\item 
		\begin{align*}
			2^{2017} \equiv 2 \times 4^{1008} \equiv 2 \times (-1)^{1008} &\equiv 2 \mod 5 \\
			2^{2017} \equiv 2 \times 64^{336} \equiv 2 \times (-1)^{336} &\equiv 2 \mod 13 \\
			2^{2017} \equiv 4 \times 32^{403} \equiv 4 \times 1^{403} &\equiv 4 \mod 31
		\end{align*}
	\par Since $2015 = 5 \times 13 \times 31$, we could apply Chinese remainder theorem to find $2^{2017} \mod 2015$.
		\begin{align*}
			?????
		\end{align*}
	
\end{enumerate}


\section*{Ex. 2 - Rabin cryptosystem}
\begin{enumerate}
	\item 
\end{enumerate}

\section*{Ex. 3 - CRT}

\end{document}