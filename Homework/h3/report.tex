\documentclass[11pt,a4paper]{article}
% \usepackage[utf8]{inputenc}
% \usepackage[T1]{fontenc}
\usepackage[english]{babel}
% \usepackage[demo]{graphicx}

% My packages
\usepackage{algorithm, algorithmic, listings} % Code
\usepackage{amsmath, amstext, amssymb, amsfonts, amsthm, dsfont, cancel, gensymb, mathtools, textcomp} % Math
\usepackage{color, xcolor} % Color
\usepackage{diagbox, tabularx} % Table
\usepackage{enumerate} % List
\usepackage{epsfig, epstopdf, graphicx, multicol, multirow, palatino, pgfplots, subcaption, tikz} % Image.
\usepackage{fancybox}
\usepackage{verbatim}

% \usepackage[font=footnotesize]{caption} % labelfont=bf
% \usepackage[font=scriptsize]{subcaption} % labelfont=bf
\usepackage[margin=1in]{geometry}
\usepackage[hidelinks]{hyperref}
\epstopdfsetup{outdir=./Figure/Converted/}
\graphicspath{{./Figure/}}

\makeatletter
\def\input@path{{./Figure/}}
\makeatother

\pgfplotsset{compat=1.13}

\def\BibTeX{{\rm B\kern-.05em{\sc i\kern-.025em b}\kern-.08em
    T\kern-.1667em\lower.7ex\hbox{E}\kern-.125emX}}

\newcommand{\image}[3]{
	\begin{figure}[!ht]
		\centering
	    \includegraphics[width=#1\textwidth]{#2}
		\caption{#3}
		\label{fig:#2}
	\end{figure}
}

\title{
	VE475 Introduction to Cryptography \\
	Homework 3
}
\author{
	Jiang, Sifan\\
	jasperrice@sjtu.edu.cn\\
	515370910040
}


\begin{document}
\maketitle


\section*{Ex. 1 - Finite fields}
\begin{enumerate}
	\item Assume $X^{2} + 1$ is reducible in $\mathbb{F}_{3}[X]$, then $X^{2} + 1$ can be written as the product of two polynomials of lower degree. The possible factors of $X^{2} + 1$ are $X$, $X + 1$, and $X + 2$.
		\begin{align*}
			X \cdot X &= X^{2}\\
			X \cdot (X + 1) &= X^{2} + X \\
			X \cdot (X + 2) &= X^{2} + 2X \\
			(X + 1) \cdot (X + 1) &= X^{2} + 2X + 1 \\
			(X + 1) \cdot (X + 2) &= X^{2} + 2 \\
			(X + 2) \cdot (X + 2) &= X^{2} + X + 1
		\end{align*}
	\par Since none of them is equal to $X^{2} + 1$, it is irreducible in $\mathbb{F}_{3}[X]$.

	\item Since $X^{2} + 1$ is irreducible in $\mathbb{F}_{3}[X]$ and $1 + 2X$ is a polynomial in $\mathbb{F}_{3}[X]$, there exists a polynomial $B(X)$ such that
		\begin{align*}
			(1 + 2X) \cdot B(X) \equiv 1 \mod (X^{2} + 1)
		\end{align*}
	\par So, $B(X)$ is the multiplicative inverse of $1 + 2X \mod X^{2} + 1$ in $F_{3}[X]$.
	
	\item Using the extended Euclidean algorithm.
	\par Initially, $r_{0}=X^{2}+1$, $r_{1}=2X+1$, $s_{0} = 0$, $s_{1} = 1$, $t_{0} = 1$, and $t_{1} = 0$.
		\begin{table}[!ht]
			\centering
			\begin{tabular}{lll|llll}
					$q$ & $r_{0}$ & $r_{1}$ & $s_{0}$ & $s_{1}$ & $t_{0}$ & $t_{1}$ \\
					\hline
					$0$ & $2X+1$ & $X^{2}+1$ & $1$ & $0$ & $0$ & $1$ \\
					$2X$ & $X+1$ & $2X+1$ & $X$ & $1$ & $1$ & $0$ \\
					$2$ & $2$ & $X+1$ & $X+1$ & $X$ & $1$ & $1$ \\
					$2X$ & $1$ & $2$ & $X^{2}+2X$ & $X+1$ & $X+1$ & $1$ \\
					$2$ & $0$ & $1$ & $X^{2}+1$ & $X^{2}+2X$ & $X+2$ & $X+1$
			\end{tabular}
		\end{table}
	\par So, the multiplicative inverse of $1+2X \mod X^{2}+1$ in $\mathbb{F}_{3}[X]$ is $X^{2}+2X = 2+2X$.
\end{enumerate}

\section*{Ex. 2 - AES}
\begin{enumerate}
	\item
	\begin{enumerate}[(a)]
		\item $\mathit{InvShiftRows}$ should be described as: cyclically shift to the right row $i$ by offset $i$, $0 \leq i \leq 3$.
		
		\item Since in the $\mathit{AddRoundKey}$ layer, each byte of the state is combined with a byte from the round key using the $\mathrm{XOR}$ operation ($\oplus$). Also, since $(a\oplus k)\oplus k = a$, where $a\in \{0, 1\}$, and $k\in \{0, 1\}$, the inverse of the layer $\mathit{AddRoundKey}$ is just applying $\mathit{AddRoundKey}$ again.
		
		\item In $\mathit{MixColumns}$ layer, we have $B = C(X) \times A$, where $B$ is the output matrix and $A$ is the state matrix. So the transformation $\mathit{InvMixColumns}$ is given by $C^{-1}(X) \times C(X) \times A = A = C^{-1}(X) \times B$. We then need to verify if the result of the multiplication of the two matrices is an identity matrix or not.
		\begin{align*}
			& \begin{pmatrix}
				00001110 & 00001011 & 00001101 & 00001001 \\
				00001001 & 00001110 & 00001011 & 00001101 \\
				00001101 & 00001001 & 00001110 &  00001011 \\
				00001011 & 00001101 & 00001001 & 00001110
			\end{pmatrix} \\
			\times &
			\begin{pmatrix}
				00000010 & 00000011 & 00000001 & 00000001 \\
				00000001 & 00000010 & 00000011 & 00000001 \\
				00000001 & 00000001 & 00000010 & 00000011 \\
				00000011 & 00000001 & 00000001 & 00000010
			\end{pmatrix} \\
			= &
			\begin{pmatrix}
				1 & 0 & 0 & 0 \\
				0 & 1 & 0 & 0 \\
				0 & 0 & 1 & 0 \\
				0 & 0 & 0 & 1
			\end{pmatrix}
			= I
		\end{align*}
		\par The detailed calculation, take the element of first row and first column as the example, is 
		\begin{align*}
			I_{0, 0} =& (00001110 \times 00000010) \oplus (00001011 \times 00000001) \oplus \\
			& (00001101 \times 00000001) \oplus (00001001 \times 00000011) \\
			=& 00011100 \oplus 00001011 \oplus 00001101 \oplus (00010010 \oplus 00001001) \\
			=& 00011100 \oplus 00001011 \oplus 00001101 \oplus 00011011 \\
			=& 1
		\end{align*}
		\par So, the matrix is the inverse matrix of $C(X)$, and the transformation $\mathit{InvMixColumns}$ is given by multiplication by the matrix.
	\end{enumerate}
	
	\item First, generate a round key according to the original $128$ bits key. Then
	
	\item
	
	\item
	\begin{enumerate}[(a)]
		\item
		
		\item
		
		\item
		
		\item
		
	\end{enumerate}
	
	\item
	
	\item
	 
\end{enumerate}

\section*{Ex. 3 - DES}
\begin{enumerate}
	\item
	
	\item
	
	\item
	
	\item
\end{enumerate}

\section*{Ex. 4 - Programming}
\end{document}
