\documentclass[11pt,a4paper]{article}
% \usepackage[utf8]{inputenc}
% \usepackage[T1]{fontenc}
\usepackage[english]{babel}
% \usepackage[demo]{graphicx}

% My packages
\usepackage{algorithm, algorithmic, listings} % Code
\usepackage{amsmath, amstext, amssymb, amsfonts, amsthm, dsfont, cancel, gensymb, mathtools, textcomp} % Math
\usepackage{color, xcolor} % Color
\usepackage{diagbox, tabularx} % Table
\usepackage{enumerate} % List
\usepackage{epsfig, epstopdf, graphicx, multicol, multirow, palatino, pgfplots, subcaption, tikz} % Image.
\usepackage{fancybox}
\usepackage{verbatim}

% \usepackage[font=footnotesize]{caption} % labelfont=bf
% \usepackage[font=scriptsize]{subcaption} % labelfont=bf
\usepackage[margin=1in]{geometry}
\usepackage[hidelinks]{hyperref}
\epstopdfsetup{outdir=./Figure/Converted/}
\graphicspath{{./Figure/}}

\makeatletter
\def\input@path{{./Figure/}}
\makeatother

\pgfplotsset{compat=1.13}

\def\BibTeX{{\rm B\kern-.05em{\sc i\kern-.025em b}\kern-.08em
    T\kern-.1667em\lower.7ex\hbox{E}\kern-.125emX}}

\newcommand{\image}[3]{
	\begin{figure}[!ht]
		\centering
	    \includegraphics[width=#1\textwidth]{#2}
		\caption{#3}
		\label{fig:#2}
	\end{figure}
}

\title{
	VE475 Introduction to Cryptography \\
	Homework 5
}
\author{
	Jiang, Sifan\\
	jasperrice@sjtu.edu.cn\\
	515370910040
}


\begin{document}
\maketitle
\section*{Ex. 1 - RSA setup}
\begin{enumerate}
\item In RSA encryption, we use
\begin{align*}
	ed \equiv 1 &\mod \varphi(n)
\end{align*}
\par Then, based on Euler's theorem, let's assume $m$ and $n$ be two coprime integers, we would have
\begin{align*}
	c^{d} \equiv m^{ed} \equiv m^{ed \mod \varphi(n)} \equiv m &\mod n,
\end{align*}
which is what expected in RSA decryption. So it is likely for $n$ to be coprime with $m$.

\item Assume $k = a\varphi(n)$, where $a \in \mathbb{N}$. % Also, assume $m < n$.
\begin{enumerate}[a)]
\item With Euler's theorem, we have
\begin{align*}
	m^{k} \equiv m^{a\varphi(n)} \equiv  (m^{\varphi(n)})^{a} \equiv 1^a \equiv 1 &\mod n \\
\end{align*}
\par Since $n = pq$, we can have $m^{k} \equiv 1 \mod p$ and $m^{k} \equiv 1 \mod q$.

\item If $\gcd(m,n) = 1$, from the result of the above question, it is obvious that $m^{k+1} \equiv m \mod p$ and $m^{k+1} \equiv m \mod q$.
\par If $\gcd(m,n) = p$, we can have $\gcd(m/p, q) = 1$, thus $(m/p)^{\varphi(q)} \equiv 1 \mod q$.
\begin{align*}
	m^{k+1} &\equiv  p \left[\left(\frac{m}{p}\right)^{k+1} \mod q\right] \mod n \\
	&\equiv p \left[\left(\frac{m}{p}\right)^{a\varphi(p)\varphi(q)+1} \mod q\right] \mod n \\
	&\equiv p \cdot \frac{m}{p} \mod n \\
	&\equiv m \mod n
\end{align*}
\par So, $m^{k} \equiv 1 \mod p$ and $m^{k} \equiv 1 \mod q$.
\par If $\gcd(m,n) = q$, similar to the case of $\gcd(m,n) = p$, we would have  $m^{k} \equiv 1 \mod p$ and $m^{k} \equiv 1 \mod q$.
\end{enumerate}

\item \begin{enumerate}[a)]
\item Since $ed \equiv 1 \mod \varphi(n)$, $ed = k + 1$. Then based on the previous result, we would have $m^{ed} \equiv m \mod n$ for all $m$.

\item Based on the previous conclusion, no matter $m$ and $n$ are coprime or not, we would have $c^{d} \equiv m^{ed} \equiv m \mod n$ in decryption. So there's no need for having $\gcd(m,n) = 1$.
\end{enumerate}
\end{enumerate}



\section*{Ex. 2 - RSA decryption}
\par We have $n = 101 \times 113$, thus $\varphi(n) = 100 \times 112 = 11200$. By applying the extended Euclidean algorithm, we can have $d = 3 = (11)_{2}$ such that $ed \equiv 1 \mod \varphi(n)$. Then the plaintext $m$ would given by calculating $c^{d} \equiv m \mod n$. We can apply modular exponentiation to find $m$.
\begin{table}[!ht]
	\centering
	\begin{tabular}{l|l|l}
		$i$ & $d_{i}$ & $power \mod 11413$ \\
		\hline
		$1$ & $1$ & $1^{2} \cdot 5859 \equiv 5859$ \\
		$0$ & $1$ & $5859^{2} \cdot 5859 \equiv 1415$
	\end{tabular}
\end{table}
\par So, $m = 1415$.



\section*{Ex. 3 - Breaking RSA}
\begin{enumerate}
\item When $d$ is small, the calculation of $c^{d} \equiv m \mod n$ with modular exponentiation would be faster.

\item Since $ed \equiv 1 \mod \varphi(n)$, we have
\begin{align*}
	ed &= K \times \varphi(pq) + 1 \\
	&= K \times \mathrm{lcm}(p-1, q-1) + 1
\end{align*}
\par Define $G = \gcd(p-1,q-1)$, the we would have
\begin{align*}
	ed = \frac{K}{G}(p-1)(q-1) + 1
\end{align*}
\par Then, define $k = \frac{K}{\gcd(K,G)}$ and $g = \frac{G}{\gcd(K,G)}$, and we would have
\begin{align*}
	ed &= \frac{k}{g}(p-1)(q-1) + 1 \\
	\frac{ed}{dpq} &= \frac{k}{dg}\frac{(p-1)(q-1)}{pq} + \frac{1}{dpq} \\
	\frac{e}{pq} &= \frac{k}{dg}(1-\delta),
\end{align*}
where $\delta = \frac{p + q - 1 - \frac{g}{k}}{pq}$. Since $p$ and $q$ are two large primes, $\delta$ would be small, then $\frac{e}{pq}$ is slightly smaller than $\frac{k}{dg}$. Also, since $ed = \frac{k}{g}(p-1)(q-1) + 1$, let $k^{*} = \dfrac{k}{g}$ we can have
\begin{align*}
	(p-1)(q-1) = \varphi(n) = \frac{ed-1}{k^{*}},
\end{align*}
where $\dfrac{e}{n}$ is slightly smaller than $\dfrac{k^{*}}{d}$. Then continued fractions is applied on $\frac{e}{pq}$ to obtain multiple approximated $\frac{k^{*}}{d}$ validate them and get the right $d$ if the equation $x^{2} - (n-\varphi(n) + 1)x + n = 0$, where $\varphi(n) = \frac{ed - 1}{k^{*}}$, has two valid solutions which are $p$ and $q$.

\item According to Wiener's theorem, if $d < \frac{1}{3}n^{\frac{1}{4}}$, the attacker can efficiently recover $d$. So, $d$ should be larger than $\frac{1}{3}n^{\frac{1}{4}}$.

\item By applying continued fractions on $\frac{e}{n}$, we have
\begin{align*}
	\frac{77537081}{317940011} = 0 + \frac{1}{4 + \frac{1}{9 + \frac{1}{1 + \frac{1}{19 + \cdots}}}}
\end{align*}
\par Then we have convergent $\frac{k^{*}}{d}$: $0, \frac{1}{4}, \frac{9}{37}, \frac{10}{41}, \frac{199}{816},\cdots$. And according to Wiener's theorem, $d < \frac{1}{3}n^{\frac{1}{4}} < 45$, we can start with $\frac{1}{4}$ and have
\begin{align*}
	(n - \varphi(n) + 1)^{2} - 4n &= (n - \frac{ed - 1}{k^{*}} + 1)^{2} - 4n = 60709145712677,
\end{align*}
which is not a square number.
\par Then try next possible $\frac{k^{*}}{d}$, and when $\frac{k^{*}}{d} = \frac{10}{41}$, we have
\begin{align*}
	(n - \varphi(n) + 1)^{2} - 4n &= (n - \frac{ed - 1}{k^{*}} + 1)^{2} - 4n = 170720356 = 13066^{2},
\end{align*}
so $p = \frac{37980 + 13066}{2} = 25523$ and $q = \frac{37980 - 13066}{2} = 12457$, thus $n = 12457 \times 25523$.
\end{enumerate}



\section*{Ex. 5 - Simple questions}
\begin{enumerate}
\item 

\item No, this double encryption isn't adding any security. The nature of breaking RSA is to factorize $n$, using double exponents won't make it more secure.

\item Since $n = 642401$, we have
\begin{align*}
	4 \cdot 516107^{2} - 187722^{2} &\equiv 0 \mod n \\
	(2 \cdot 516107 - 187722)(2 \cdot 516107 + 187722) &\equiv 0 \mod n \\
	844492 \cdot 1219936 &\equiv 0 \mod n \\
	(-440310) \cdot (-64866) &\equiv0 \mod n \\
	(2 \cdot 3 \cdot 5 \cdot 13 \cdot 1129) \cdot (2 \cdot 3 \cdot 19 \cdot 569) &\equiv 0 \mod n
\end{align*}
\par So, we would get $n = 642401 = 569 \times 1129$.

\item With three primes $p$, $q$, and $r$, $n = pqr$ and $\varphi(n) = (p-1)(q-1)(r-1)$. Find $e$ such that $\gcd(e, \varphi(n)) = 1$, and then find $d$ such that
\begin{align*}
	ed \equiv 1 \mod \varphi(n)
\end{align*}
\par Then
\begin{align*}
c &\equiv m^{e} \mod n \\
m &\equiv c^{d} \equiv m^{ed} \equiv m^{\varphi(n) + 1} \mod n
\end{align*}
\par If the length of the public keys are same in both cases, it would result in short separate primes, making the factorization easier.

\item Since $97 - 1 = 96 = 2^{5} \times 3$, then a $\alpha$ is a generator if $\alpha^{48} \not\equiv 1 \mod 97$ and $\alpha^{32} \not\equiv 1 \mod 97$. Or, since $32 = 2 \times 16$ and $48 = 3 \times 16$, we can first calculate $2$
\begin{align*}
	\alpha^{16} \not\equiv \pm 1, 35, 61 \mod 97
\end{align*}
\par We can take the numbers in consequence and have
\begin{align*}
	2^{16} &\equiv 61 \mod 97 \\
	3^{16} &\equiv 61 \mod 97 \\
	4^{16} &\equiv 1  \mod 97 \\
	5^{16} &\equiv 36 \mod 97
\end{align*}
\par So, $5$ is the smallest generator of the group.

\item \begin{enumerate}[a)]
\item Since $101 - 1 = 100 = 2^{2} \times 5^{2}$, we would have
\begin{align*}
	2^{\frac{100}{2}} \equiv 2^{50} \equiv 100 \not\equiv \mod 101
\end{align*}
Also,
\begin{align*}
	2^{\frac{100}{5}} \equiv 2^{20} \equiv 95 \not\equiv 1 \mod 101
\end{align*}
\par So, $2$ is a generator of $G$.

\item Since $\log_{2}2 = 1$, we would have
\begin{align*}
	\log_{2}24 = \log_{2}3 + 3\log_{2}2 = 69 + 3 = 72
\end{align*}

\item Since in group $G$,
\begin{align*}
	\log_{2}24 = \log_{2}(24 + 101) = \log_{2}(125) = 3\log_{5} = 3 \times 24 = 72
\end{align*}
\end{enumerate}

\item Since $\gcd(3, 137) = 1$, we have $3^{\varphi(137)} \equiv 3^{136} \equiv \mod 137$. Also, notice that $44 = 2^{2} \times 11$, we would have
\begin{align*}
	3^{6} \equiv 3^{136 + 6} \equiv 3^{142} \equiv 44 \equiv 2^{2} \times 11 \equiv (3^{10})^{2} \times 3^{x} \mod 137
\end{align*}
\par So, $x = 142 - 2 \times 10 = 122$.

\item \begin{enumerate}[a)]
\item Since $6^{5} \equiv 10 \mod 11$, $6^{5}$ in $G$ is $10$.

\item For $q \vert (p-1)$, $q \in \{2, 5\}$.
\begin{align*}
	2^{\frac{10}{2}} \equiv 10 \not\equiv 1 &\mod 11 \\
	2^{\frac{10}{5}} \equiv 4 \not\equiv 1 &\mod 11
\end{align*}
\par So $2$ is a generator of $G$.

\item From previous result, we have
\begin{align*}
	2^{5x} \equiv 10^{x} \equiv (-1)^{x} \mod 11
\end{align*}
\par Also,
\begin{align*}
	2^{5x} \equiv 6^{5} \equiv -1 \mod 11
\end{align*}
\par So, $(-1)^{x} = -1$, thus $x$ is odd.
\end{enumerate}
\end{enumerate}



\section*{Ex. 6 - DLP}
\begin{enumerate}
\item From what we known, we can have
\begin{align*}
	3^{x} &\equiv 2 \mod 65537 \\
	3^{16x} &\equiv -1 \mod 65537 \\
	3^{32x} &\equiv 1 \mod 65537
\end{align*}
\par Since $3$ and $65537$ are coprime integers and $\varphi(65537) = 65536$, we would also have
\begin{align*}
	3^{65536} \equiv 1 \mod 65537
\end{align*}
\par So $65536 \vert 32x$ and $65536 \nmid 16x$, which gives that $2048$ divides $x$, while $4096$ does not.

\item Let $x = (2k + 1)\cdot 2048$, where $k \in \mathbb{N}$. And there are $16$ possible choices for $k$, which are $0, 1, 2, \cdots, 15$. And when $k = 13$ ($x = 55296$), we have $3^{x} \equiv 2 \mod 65537$.

\item Since $x\vert 2048$ and $x \nmid 4096$, we can apply the Pohlig-Hellman algorithm by using
\begin{align*}
	x = 2^{11} + a_{12}2^{12} + a_{13}2^{13} + a_{14}2^{14} + a_{15}2^{15}
\end{align*}
\par For $a_{12}$,
\begin{align*}
	\left(\frac{3^{x}}{3^{2^{11}}}\right)^{2^{15-12}} \equiv (2^{14})^8 \equiv -1 \mod 65537
\end{align*}
\par So, $a_{12} = 1$.
\par For $a_{13}$,
\begin{align*}
	\left(\frac{3^{x}}{3^{2^{11}+2^{12}}}\right)^{2^{15-13}} \equiv (2^{8})^4 \equiv 1 \mod 65537
\end{align*}
\par So, $a_{13} = 0$.
\par Similarly, we would have $a_{14} = 1$ and $a_{15} = 1$, which gives
\begin{align*}
	x = 2^{11} + 2^{12} + 2^{14} + 2^{15} = 55296.
\end{align*}

\item $65537$ is a prime but in the form $p^{k} + 1$. If $c^{x} \equiv p \mod p^{k} + 1$, in order to find $x$, we can find a generator of the group and since $c^{2k} \equiv p^{2k} \equiv 1 \mod p^{k} + 1$, we can find $\frac{p^{k}}{2k} \vert x$ and $\frac{p^{k}}{k} \nmid x$. Then there would only be $k$ possible choices for $x$.
\end{enumerate}
\end{document}