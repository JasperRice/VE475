\documentclass[11pt,a4paper]{article}
% \usepackage[utf8]{inputenc}
% \usepackage[T1]{fontenc}
\usepackage[english]{babel}
% \usepackage[demo]{graphicx}

% My packages
\usepackage{algorithm, algorithmic, listings} % Code
\usepackage{amsmath, amstext, amssymb, amsfonts, amsthm, dsfont, cancel, gensymb, mathtools, textcomp} % Math
\usepackage{color, xcolor} % Color
\usepackage{diagbox, tabularx} % Table
\usepackage{enumerate} % List
\usepackage{epsfig, epstopdf, graphicx, multicol, multirow, palatino, pgfplots, subcaption, tikz} % Image.
\usepackage{fancybox}
\usepackage{verbatim}

\usepackage[font=footnotesize]{caption} % labelfont=bf
\usepackage[font=scriptsize]{subcaption} % labelfont=bf
\usepackage[margin=1in]{geometry}
\usepackage[hidelinks]{hyperref}
\epstopdfsetup{outdir=./Figure/Converted/}
\graphicspath{{./Figure/}}

\makeatletter
\def\input@path{{./Figure/}}
\makeatother

\pgfplotsset{compat=1.13}

\newcommand{\image}[3]{
	\begin{figure}[!ht]
		\centering
	    \includegraphics[width=#1\textwidth]{#2}
		\caption{#3}
		\label{fig:#2}
	\end{figure}
}

\title{
	Homework 1\\
	VE475 Introduction to Cryptography
}
\author{
	Jiang, Sifan\\
	jasperrice@sjtu.edu.cn\\
	515370910040
}


\begin{document}
\maketitle

\section*{Ex. 1 - Simple questions}
\begin{enumerate}
	\item Since Alice uses Caesar cipher and we already known the ciphertext, we can apply CCA to decrypt the ciphertext, ``EVIRE''. So, the $25$ possible names of the place are: \textit{duhqd}, \textit{ctgpc}, \textit{bsfob}, \textit{arena}, \textit{zqdmz}, \textit{ypcly}, \textit{xobkx}, \textit{wnajw}, \textit{vmziv}, \textit{ulyhu}, \textit{tkxgt}, \textit{sjwfs}, \textit{river}, \textit{qhudq}, \textit{pgtcp}, \textit{ofsbo}, \textit{neran}, \textit{mdqzm}, \textit{lcpyl}, \textit{kboxk}, \textit{janwj}, \textit{izmvi}, \textit{hyluh}, \textit{gxktg}, or \textit{fwjsf}. However, only ``\textit{arena}'' and ``\textit{river}'' are meaningful, which could be the meeting place.
	
	\item Since the length of plaintext \textit{dont} is $4$, reasonable size of the key should be $2\times 2$. Label letters as integers from $0$ to $25$, the plaintext then is $\begin{pmatrix} 3 & 14 & 13 & 19 \end{pmatrix}$ and the ciphertext is $\begin{pmatrix} 4 & 11 & 13 & 8 \end{pmatrix}$. After splitting the letters, we would have
	\begin{align*}
		\underbrace{\begin{pmatrix} 3 & 14 \\ 13 & 19 \end{pmatrix}}_{A} \cdot \begin{pmatrix} a & b \\ c & d \end{pmatrix} \equiv \begin{pmatrix} 4 & 11 \\ 13 & 8 \end{pmatrix} \mod 26 
	\end{align*}
	\par Since $\det(A)=-125$ and $\gcd(-125, 26)=1$, $A$ is invertible modulo $26$. Also, we can obtain that $-5$ is the multiplicative inverse of $-125$ modulo $26$ by applying extended Euclidean algorithm. We can then calculate
	\begin{align*}
		K = \begin{pmatrix} a & b \\ c & d \end{pmatrix} &\equiv \begin{pmatrix} 3 & 14 \\ 13 & 19 \end{pmatrix}^{-1} \cdot \begin{pmatrix} 4 & 11 \\ 13 & 8 \end{pmatrix} \mod 26 \\
		&\equiv \frac{1}{-125} \cdot \begin{pmatrix} 19 & -14 \\ -13 & 3 \end{pmatrix} \cdot \begin{pmatrix} 4 & 11 \\ 13 & 8 \end{pmatrix} \mod 26 \\
		&\equiv \frac{1}{-125} \cdot \begin{pmatrix} -106 & 97 \\ -13 & -119 \end{pmatrix} \mod 26 \\
		&\equiv \begin{pmatrix} 530 & -485 \\ 65 & 595 \end{pmatrix} \mod 26 \\
		&\equiv \begin{pmatrix} 10 & 9 \\ 13 & 23 \end{pmatrix} \mod 26
	\end{align*}
	\par So, the encryption matrix is $K = \begin{pmatrix} 10 & 9 \\ 13 & 23 \end{pmatrix}$.
	
	\item We can suppose that $n | b$ is wrong, which gives $b = kn + d$, where $k$ and $d$ are integers and $d \in (0, n)$. Since $n|ab$, we would have $ab = ln$, where $l$ is an integer. After combining two relations above, we have
	\begin{align*}
		a(kn + d) &= ln \\
		akn + ad &= ln \\
		ad &= (l - ak)n \\
		\frac{ad}{n} &= l - ak
	\end{align*}
	\par Since $l - ak$ is an integer, so should $\frac{ad}{n}$ be an integer. However, since $\gcd(a, n) = 1$ and $d \in (0, n)$, which gives $n \nmid a$ and $n \nmid d$, $\frac{ad}{n}$ could not be an integer. So, the assumption is wrong and $n | b$.
	
	\item Applying Euclidean algorithm
	\begin{align*}
		30030 &= 116 \times 257 + 218 \\
		257 &= 1 \times 218 + 39 \\
		218 &= 5 \times 39 + 23 \\
		39 &= 1 \times 23 + 16 \\
		23 &= 1 \times 16 + 7 \\
		16 &= 2 \times 7 + 2 \\
		7 &= 3 \times 2 + 1 \\
		2 &= 2 \times 1
	\end{align*}
	\par So, $\gcd(30030, 257) = 1$.
	\par Since $16^{2} = 256 < 257 < 289 = 17^{2}$, the factor of $257$ could only be obtained from $2$, $3$, $5$, $7$, $11$, and $13$. However, none of these primes can exact divide $257$. So, $257$ is a prime.
	
	\item Once Eve got the key and if the same key is used to encrypt the plaintext, Eve can easily break the ciphertext by using the key to XOR the ciphertext, which can be seen from tab~\ref{tab:XOR}.
	\begin{table}[!ht]
		\caption{Truth table.}
		\label{tab:XOR}
		\centering
		\begin{tabular}{|c|c|c|c|}
			\hline 
			$a$ & $k$ & $a^{\wedge}k$ & $(a^{\wedge}k)^{\wedge}k$ \\ 
			\hline 
			1 & 1 & 0 & 1 \\ 
			\hline 
			1 & 0 & 1 & 1 \\ 
			\hline 
			0 & 1 & 1 & 0 \\ 
			\hline 
			0 & 0 & 0 & 0 \\ 
			\hline 
		\end{tabular}
	\end{table}
	
	\item Being secure meaning it should take at least $2^{128}$ operations to break the encryption. So
	\begin{align*}
		\sqrt{n\log n} &\geq 128 \\
		n &\geq 4486.4
	\end{align*}
	\par So, the size of $4487$ (or larger) should be used for the graph to be secure.
\end{enumerate}


\section*{Ex. 2 - Vigen\`{e}re cipher}
\begin{enumerate}
	\item 
	
	\item 
\end{enumerate}
	
\end{document}